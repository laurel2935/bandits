\section{Introduction}
There are a number of natural bandit problems that arise for which the
number of possible arms is infinite.  For example, consider the problem of
sensor placement for maximum coverage -- in some given area there can be an
infinite number of locations to place a sensor, and with multiple sensors the
problem only becomes even higher-dimensional.  Alternatively, consider the
problem of determining the optimal parameter of some physical system, where
this parameter is simply a real number in some range.  In both cases,
evaluating a reward function can be quite costly, and as we have limited
feedback, this is a bandit problem.  However, the standard finite-armed
bandit algorithms are not suitable when there are an infinite number of
possible arms to choose, and so we must turn to other methods in order to
solve these problems.

One way to handle the infinite-armed
case is to simply discretize whatever domain our bandit problem is over
into some finite number of points, then run a standard finite-armed bandit
algorithm on those points.  However, this does not account for any
possible correlation between the points that were chosen -- i.e. any possible
information gain from reward function smoothness is lost.  With that in
mind, bandit algorithms have been developed particularly for the case in
which there are an infinite number of arms.  Here we present two such
algorithms, the Hierarchical Optimistic Optimization algorithm and the
Zooming algorithm, and then we present an analysis of how these algorithms
compare with various discretization algorithms over several test cases.
