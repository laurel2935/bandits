\newcommand{\OPT}{\textsf{OPT}}
\newcommand{\NULL}{\textsf{null}}

% ==============================================
% TEXT MANIPULATION MACROS, BASIC MACROS
% ==============================================
\newcommand{\ignore}[1]{}
\newcommand{\myskip}{\myvspace{-1mm}}
\newcommand{\captionsmall}[1]{\caption{\small #1}}
\newcommand{\commentout}[1]{}

\newcommand{\cf}{\emph{cf.}}
\def \etal {\emph{et al.}\ }

\newcommand{\ith}{\ensuremath{i^{\mathrm{th}}}\xspace}
\newcommand{\jth}{\ensuremath{j^{\mathrm{th}}}\xspace}
\newcommand{\tth}{\ensuremath{t^{\mathrm{th}}}\xspace}



% ==============================================
% REFERENCES
% ==============================================
\newcommand{\citet}[1]{\cite{#1}}
\newcommand{\citep}[1]{\cite{#1}}
\newcommand{\citecf}[1]{(\cf, \cite{#1})}

\newcommand{\tableref}[1]{Tab.~\ref{#1}}
\newcommand{\figref}[1]{Fig.~\ref{#1}}
\newcommand{\vfigref}[1]{Fig.~\vref{#1}}
\newcommand{\fullfigref}[1]{Fig.~\ref{#1} on page~\pageref{#1}}
\newcommand{\eqnref}[1]{Eq.~(\ref{#1})}
\newcommand{\secref}[1]{Sec.~\ref{#1}}
\newcommand{\thmref}[1]{Theorem~\ref{#1}}
\newcommand{\propref}[1]{Prop.~\ref{#1}}
\newcommand{\lemref}[1]{Lemma~\ref{#1}}
\newcommand{\corref}[1]{Corollary~\ref{#1}}
\newcommand{\algref}[1]{Alg.~\ref{#1}}
\newcommand{\lineref}[1]{Line~\ref{#1}}
\newcommand{\probref}[1]{Problem~(\ref{#1})}


% ==============================================
% LIST ENVIRONMENTS
% ==============================================

\newcommand{\denselist}{
    \itemsep -3pt\topsep-8pt\partopsep-8pt
}

\newcommand{\denselistbib}{
    \itemsep 0pt\topsep-4pt\partopsep-5pt
}

\newenvironment{Enum}{
\vspace{-0.5em}
\begin{enumerate}
   \setlength{\itemsep}{0.25em}%{3pt}
  \setlength{\parskip}{0em}
  \setlength{\parsep}{0em}}
{\end{enumerate}\vspace{-0.5em}}

\newcounter{myLISTctr}
\newcommand{\initOneLiners}{%
    \setlength{\itemsep}{0pt}
    \setlength{\parsep }{0pt}
    \setlength{\topsep }{0pt}
%      \usecounter{myLISTctr}
}
\newenvironment{OneLiners}[1][\ensuremath{\bullet}]
    {\begin{list}
        {#1}
        {\initOneLiners}}
    {\end{list}}

% ==============================================
% COMMON SETS, OBJECTS, and VARIABLES
% ==============================================
\newcommand{\integers}{\ensuremath{\mathbb{Z}}}
\newcommand{\nats}{\ensuremath{\mathbb{N}}}
\newcommand{\reals}{\ensuremath{\mathbb{R}}}
\newcommand{\Real}{\mathbb R}
\newcommand{\NonNegativeReals}{\ensuremath{\mathbb{R}_{\ge 0}}}

\newcommand{\eps}{\varepsilon}
\newcommand{\distrib}{{\mathcal{D}}}
\newcommand{\family}{{\mathcal{F}}}

% complexity classes
\newcommand{\class} [1] {\textrm{#1}} % bf or sc or rm?
\renewcommand{\P} {\class{P}}
\newcommand{\NP} {\class{NP}}

% ===================================================
% COMMONLY USED OPERATORS, FUNCTIONS, PARENS, ARROWS
% ===================================================
\def \argmax {\mathop{\rm arg\,max}}
\def \argmin {\mathop{\rm arg\,min}}


\newcommand{\paren} [1] {\ensuremath{ \left( {#1} \right) }}
\newcommand{\curlyparen} [1] {\ensuremath{ \left\{ {#1} \right\} }}
\newcommand{\bigparen} [1] {\ensuremath{ \Big( {#1} \Big) }}
\newcommand{\set}[1]{\ensuremath{\left\{#1\right\}}}
\renewcommand{\Pr}[1]{\ensuremath{\mathbb{P}\left[#1\right] }}
\DeclareMathOperator{\pr}{Pr} 
\newcommand{\prob}[1]{\ensuremath{\mathbf{Pr}\left[#1\right] }}
\DeclareMathOperator{\dist}{Dist}
\DeclareMathOperator{\dom}{dom}

\newcommand{\norm}[1]{\left\Vert#1\right\Vert}
\newcommand{\abs}[1]{\left\vert#1\right\vert}

% Expectations
\newcommand{\expt}[1]{\mathbb{E}\left[#1\right]}
\newcommand{\E}[1]{\ensuremath{\mathbb{E}\left[#1\right] }}
\newcommand{\Eunder}[2]{\ensuremath{\mathbb{E}_{#1}\left[#2\right] }}
\newcommand{\Econd}[2]{\ensuremath{\mathbb{E}\left[  {#1} \ \middle| \ {#2}\right] }}  % conditional expectation

\DeclareMathOperator*{\Var}{Var}
%\newcommand{\abs}[1]{\ensuremath{\left|#1\right|}}
\newcommand{\size}[1]{\ensuremath{\left|#1\right|}}
\newcommand{\ceil}[1]{\ensuremath{\left\lceil#1\right\rceil}}
\newcommand{\floor}[1]{\ensuremath{\left\lfloor#1\right\rfloor}}
%\newcommand{\norm}[2]{\ensuremath{\Vert {#1} \Vert_{#2}}}
\newcommand{\tuple}[1]{\ensuremath{\langle #1 \rangle}}
\newcommand{\func}[3]{\ensuremath{#1 : #2 \rightarrow #3}}
\newcommand{\poly}{\operatorname{poly}}

\newcommand{\To}{\longrightarrow}

% ==============================================
% ASYMPTOTICS
% ==============================================
\newcommand{\littleO}[1]{\ensuremath{o\paren{#1}}}
\newcommand{\bigO}[1]{\ensuremath{O\paren{#1}}}
\newcommand{\bigTheta}[1]{\ensuremath{\Theta\paren{#1}}}
\newcommand{\bigOmega}[1]{\ensuremath{\Omega\paren{#1}}}

% ==============================================
% PSEUDOCODE (these require the package algorithm2e)
% ==============================================
 \SetKw {KwEach} {\hspace{-1mm}each}
\SetKw {KwForEach} {for each}
\SetKw {KwFrom} {from}
\SetKw {KwSet} {set}
\SetKw {KwReturn} {return}
\SetKw {KwIf} {if}
\SetKw {KwThen} {then}
% ==============================================
% MISCELLANEOUS
% ==============================================
\renewcommand{\paragraph}[1]{\vspace{3mm}\noindent{\bf #1.} }



% ========= Andreas' Macros  ========================
% ----------------------------------------------------------------
\vfuzz2pt % Don't report over-full v-boxes if over-edge is small
\hfuzz2pt % Don't report over-full h-boxes if over-edge is smallq
% THEOREMS -------------------------------------------------------
%\newtheorem{thm}{Theorem}%[section]
\newtheorem{lemma}{Lemma}
\newtheorem{theorem}{Theorem}
\newtheorem{cor}[theorem]{Corollary}
%\newtheorem{lem}[thm]{Lemma}
\newtheorem{prop}[theorem]{Proposition}
\newtheorem{problem}[theorem]{Problem}
\newtheorem{example}[theorem]{Example}
%\theoremstyle{definition}
\newtheorem{defn}[theorem]{Definition}
%\theoremstyle{remark}
%\newtheorem{rem}[thm]{Remark}
%\numberwithin{equation}{section}

%-------------------------------------------------------------
%                       subfigure
%-------------------------------------------------------------

%% enable subfigures
%\usepackage{subfigure}

%\newbox\subfigbox
%\makeatletter
%  \newenvironment{subfloat}
%    {\def\caption##1{\gdef\subcapsave{\relax##1}}%
%     \let\subcapsave\@empty
%     \setbox\subfigbox\hbox
%       \bgroup}
%      {\egroup
%     \subfigure[\subcapsave]{\box\subfigbox}}
%\makeatother


% ---------------


% MISC OTHER MATH STUFF ---------------------
\newcommand{\cS}{{\mathcal{S}}}
\newcommand{\cT}{{\mathcal{T}}}
\newcommand{\cA}{{\mathcal{A}}}
\newcommand{\cR}{{\mathcal{R}}}
\newcommand{\cP}{{\mathcal{P}}}
\newcommand{\cV}{{\mathcal{V}}}
\newcommand{\cB}{{\mathcal{B}}}
\newcommand{\cX}{{\mathcal{X}}}
\newcommand{\cU}{{\mathcal{U}}}
\newcommand{\cW}{{\mathcal{W}}}
\newcommand{\cC}{{\mathcal{C}}}
\newcommand{\cD}{{\mathcal{D}}}
\newcommand{\cI}{{\mathcal{I}}}

\newcommand{\balpha}{{\mathbf{\alpha}}}
\newcommand{\ba}{{\mathbf{a}}}
\newcommand{\bh}{{\mathbf{h}}}
\newcommand{\bs}{{\mathbf{s}}}
\newcommand{\bt}{{\mathbf{t}}}
\newcommand{\bu}{{\mathbf{u}}}
\newcommand{\bv}{{\mathbf{v}}}
\newcommand{\bw}{{\mathbf{w}}}
\newcommand{\bx}{{\mathbf{x}}}
\newcommand{\by}{{\mathbf{y}}}
\newcommand{\bz}{{\mathbf{z}}}
\newcommand{\bZ}{{\mathbf{Z}}}

\newcommand{\hF}{{\widehat{F}}}
\newcommand{\sF}{\mathcal{\bar{F}}}
\newcommand{\bF}{{\overline{F}}}

\DeclareMathOperator{\kernel}{\mathcal{K}}
\DeclareMathOperator{\meanf}{\mathcal{M}}


